\newpage

\thispagestyle{empty}

{
\setlength{\parindent}{0em}
\renewcommand{\baselinestretch}{2}
{\songti\sihao\bfseries

一、 \; 题目: \; \underline{\makebox[24em]{\zjutitlec}}

\vspace{2em}

二、 \; 指导教师对文献综述和开题报告的具体内容要求: \\ \par
}
{
  \songti\xiaosi 

% 多个摄像头下的行人再识别是一个非常挑战的问题,尤其是摄像头之间没有交叉视野的情况下。现有的算法主要集中在使用相同的深度神经网络对输入的不同视角下的图像或视频对,进行特征学习,然后通过距离度量,确定是否代表同一个人。但是,一方面由于受到标注数据集有限的影响,算法的性能会受到影响;另一方面,现有的算法不能很好的利用图像和视频数据的特征多样性,提升算法的鲁棒性。 研究内容:在本次毕业论文中,一方面,我们计划引入生成对抗的思想,通过产生逼真的假样本,对训练的样本集进行扩充,提高算法的总体性能;另一方面,利用多种深度网络构建多种特征来学习样本之间的相似性,增强算法的鲁棒性。 具体要求:学会分析图像视频数据,构建新颖的模型,解决现有算法的不足,争取发表一篇高水平学术论文。

文献综述的主要目的是为论文的写作和研究方法提供支持, 进而表明本研究是有价值的。具体要求归纳和总结国内外相关文献对该研究问题做过哪些方面的研究,取得了哪些阶段性的成果,以及既有研究中存在的一些难点和局限性。在进行文献综述时,不要只是简单罗列之前的研究成果,而是要紧扣自己的研究内容,对与自己要解决的具体研究问题紧密相关的文献进行必要的归纳、总结、批判并提出一些新的观点和可能的新研究方向。开题报告一般由以下几个方面组成:1)研究问题;2)研究目的和意义;3)文献综述;4)研究方法;5)论文的纲要;6)参考文献。
}

\vspace{9cm}
}

{
\songti\xiaosi\bfseries
\begin{flushright}
  指导教师(签名) \; \underline{\hspace{6em}} \\
  年 \qquad 月 \qquad 日
\end{flushright}
}

\ifthenelse{\equal{\zjuside}{T}}{\newpage\mbox{}\thispagestyle{empty}}{}
