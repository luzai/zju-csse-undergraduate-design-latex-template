% !TeX spellcheck = en_US
% !TeX encoding = utf8
% !TeX program = xelatex
% !BIB program = biber 

% \documentclass{beamer}
\documentclass[notes]{beamer}
% \documentclass[draft]{beamer}	
% \usetheme{Singapore}
% \usetheme{Hannover}
\usepackage{pgfpages}
\setbeameroption{hide notes} % Only slides
% \setbeameroption{show only notes} % Only notes
% \setbeameroption{show notes on second screen=right} % Both

\usepackage[british]{babel}
\usepackage{graphicx,hyperref,url, booktabs}
% \usepackage{ru}
\usepackage{math}
% \usepackage{hanging}
\usepackage{listings}
\usefonttheme[onlymath]{serif}
\usepackage{fontspec,xunicode}
% \setmainfont{Tahoma}
\usepackage[slantfont,boldfont]{xeCJK}
\setmainfont{Times New Roman}%缺省英文字体.serif是有衬线字体sans serif无衬线字体
\setCJKmainfont[ItalicFont={Adobe Kaiti Std}, BoldFont={Adobe Heiti Std}]{STSong}%衬线字体 缺省中文字体为
\setCJKsansfont{STSong}
\setCJKmonofont{STFangsong}%中文等宽字体

\pgfdeclareimage[width=\paperwidth,height=\paperheight]{bg}{background}
\setbeamertemplate{background}{\pgfuseimage{bg}}

\usepackage{animate}
% \usepackage[round]{natbib}
% \bibliographystyle{plainnat}
% \bibliographystyle{apalike} 
% \usepackage[backend=biber]{biblatex}
\usepackage{biblatex}

\bibliography{./ref.bib}
\addbibresource{ref.bib}
\usepackage{multirow}
\usepackage{booktabs}
\usepackage{indentfirst}
\usepackage{longtable}
\usepackage{float}
% \usepackage{picins}
\usepackage{rotating}
\usepackage{subfigure}
\usepackage{tabu}
\usepackage{amsmath}
\usepackage{amssymb}
\usepackage{setspace}
\usepackage{amsfonts}
\usepackage{appendix}
\usepackage{listings}
\usepackage{xcolor}
\usepackage{geometry}

%%-----------------------xeCJK下设置中文字体------------------------------%
\setCJKfamilyfont{song}{SimSun}                             %宋体 song
\newcommand{\song}{\CJKfamily{song}}
\setCJKfamilyfont{fs}{FangSong_GB2312}                      %仿宋2312 fs
\newcommand{\fs}{\CJKfamily{fs}}
\setCJKfamilyfont{yh}{Microsoft YaHei}                    %微软雅黑 yh
\newcommand{\yh}{\CJKfamily{yh}}
\setCJKfamilyfont{hei}{SimHei}                              %黑体  hei
\newcommand{\hei}{\CJKfamily{hei}}
\setCJKfamilyfont{hwxh}{STXihei}                                %华文细黑  hwxh
\newcommand{\hwxh}{\CJKfamily{hwxh}}
\setCJKfamilyfont{asong}{Adobe Song Std}                        %Adobe 宋体  asong
\newcommand{\asong}{\CJKfamily{asong}}
\setCJKfamilyfont{ahei}{Adobe Heiti Std}                            %Adobe 黑体  ahei
\newcommand{\ahei}{\CJKfamily{ahei}}
\setCJKfamilyfont{akai}{Adobe Kaiti Std}                            %Adobe 楷体  akai
\newcommand{\akai}{\CJKfamily{akai}}

\newcommand{\verylarge}{\fontsize{60pt}{\baselineskip}\selectfont}  
\newcommand{\chuhao}{\fontsize{44.9pt}{\baselineskip}\selectfont}  
\newcommand{\xiaochu}{\fontsize{38.5pt}{\baselineskip}\selectfont}  
\newcommand{\yihao}{\fontsize{27.8pt}{\baselineskip}\selectfont}  
\newcommand{\xiaoyi}{\fontsize{25.7pt}{\baselineskip}\selectfont}  
\newcommand{\erhao}{\fontsize{23.5pt}{\baselineskip}\selectfont}  
\newcommand{\xiaoerhao}{\fontsize{19.3pt}{\baselineskip}\selectfont} 
\newcommand{\sihao}{\fontsize{14pt}{\baselineskip}\selectfont}      % 字号设置  
\newcommand{\xiaosihao}{\fontsize{12pt}{\baselineskip}\selectfont}  % 字号设置  
\newcommand{\wuhao}{\fontsize{10.5pt}{\baselineskip}\selectfont}    % 字号设置  
\newcommand{\xiaowuhao}{\fontsize{9pt}{\baselineskip}\selectfont}   % 字号设置  
\newcommand{\liuhao}{\fontsize{7.875pt}{\baselineskip}\selectfont}  % 字号设置  
\newcommand{\qihao}{\fontsize{5.25pt}{\baselineskip}\selectfont}    % 字号设置 

\graphicspath{{./fig/}}

% \setbeamertemplate{footnote}{%
%   \hangpara{2em}{1}%
%   \makebox[2em][l]{\insertfootnotemark}\footnotesize\insertfootnotetext\par%
% }

\definecolor{cred}{rgb}{0.6,0,0}
\definecolor{cgreen}{rgb}{0.25,0.5,0.35}
\definecolor{cpurple}{rgb}{0.5,0,0.35}
\definecolor{cdocblue}{rgb}{0.25,0.35,0.75}
\definecolor{cdark}{rgb}{0.95,1.0,1.0}
\lstset{
	language=python,
	numbers=left,
	numberstyle=\tiny\color{black},
	keywordstyle=\color{cpurple},
	commentstyle=\color{cgreen},
	stringstyle=\color{cred},
	% frame=single,
	% escapeinside=``,
	% xleftmargin=1em,
	% xrightmargin=1em, 
	backgroundcolor=\color{cdark},
	% aboveskip=1em,
	% breaklines=true,
	% tabsize=3
} 

\makeatletter
\long\def\beamer@author[#1]#2{%
  \def\insertauthor{\def\inst{\beamer@insttitle}\def\and{\beamer@andtitle}%
  \begin{tabular}{rl}#2\end{tabular}}%
  \def\beamer@shortauthor{#1}%
  \ifbeamer@autopdfinfo%
    \def\beamer@andstripped{}%
    \beamer@stripands#1 \and\relax
    {\let\inst=\@gobble\let\thanks=\@gobble\def\and{: }\hypersetup{pdfauthor={\beamer@andstripped}}}
  \fi%
}
\makeatother

% \setbeamersize{text margin left=60mm} 
\setbeamertemplate{frametitle}[default][right]

% The title of the presentation:
%  - first a short version which is visible at the bottom of each slide;
%  - second the full title shown on the title slide;
\title[毕设答辩]{\ahei   毕业论文答辩}

% Optional: a subtitle to be dispalyed on the title slide
\subtitle{\akai  深度行人再识别学习}

% The author(s) of the presentation:
%  - again first a short version to be displayed at the bottom;
%  - next the full list of authors, which may include contact information;
\author[xinglu]{
	姓名学号: & 王兴路 3140102282 \\
	指导老师: & 李英明 \\
	年级专业: & 2014级信息工程
}
% The institute:
%  - to start the name of the university as displayed on the top of each slide
%    this can be adjusted such that you can also create a Dutch version
%  - next the institute information as displayed on the title slide

\institute[信工1403]{}

% Add a date and possibly the name of the event to the slides
%  - again first a short version to be shown at the bottom of each slide
%  - second the full date and event name for the title slide
\date[\today]{2018年3月16日}

\begin{document}

\AtBeginSection[]
{
	\begin{frame}
		\frametitle{大纲}
		\tableofcontents[currentsection]
	\end{frame}
}

\AtBeginSubsection[2-]
{
   \begin{frame}
       \frametitle{大纲}
       \tableofcontents[currentsection]
   \end{frame}
}

\begin{frame}
	\titlepage
	\note{2018/06/01--2018/06/08毕业论文答辩。}
\end{frame}

% \begin{frame}{Embedded Animation}
%   \animategraphics[loop,controls,width=.6\linewidth]{10}{stn-}{0}{52}
% \end{frame}

\section{背景介绍与研究内容}

\begin{frame}
	{背景介绍}
	\begin{itemize}
		\item 行人再识别在智能视频监控、智能安防邻域应用广泛
		\item 摄像机网络已广泛布控于各种公共场合中,急需{\hei 跨摄像头}检索行人的技术
		\item 摄像头采集了{\hei 海量}数据,对海量数据进行压缩,建立索引并快速查找
	\end{itemize}
	\begin{figure}
		\includegraphics[width=0.75\linewidth]{2018-03-07-19-33-13.png}
	\end{figure}
	\note{
		随着监控摄像机的大规模安装,摄像机网络已广泛布控于各种公共场合中。

		利用摄像机网络,我们可以追踪同一行人的轨迹.

		但是面对多
		路监控和海量的视频数据,监控人员很容易 疲惫和应接不暇 ,我们需要智能分析技术 来完成海量的工作
	}
\end{frame}


\begin{frame}
	{行人再识别定义}
	\begin{block}{前提假设}
		\begin{itemize}
			\item 检测器已经成功检测
			\item \ie, 再识别{\bf 不包括}检测步骤
		\end{itemize}
		% 输入某一行人的询问图片(Probe), 在图像或者视频集合(gallery)中跨摄像头检索所有包含该行人的图片. \cite{zheng2016person},\cite{Zheng2017person}
	\end{block}
	\begin{figure}
		\includegraphics[width=1\linewidth]{background.png}
	\end{figure}
	\note{
		目前学术界几乎所有论文都采用这样的前提假设。行人再识别不包括行人检测的步骤。
		我们不讨论行人再识别的定义是否有问题,先遵循普遍接受的定义
	}
\end{frame}



\begin{frame}
	{行人再识别定义}
	\begin{block}{行人再识别 (Person Re-identification, \textit{aka.} ReID)}
		输入某一行人的询问图片(Probe), 在图像或者视频集合(gallery)中跨摄像头检索所有包含该行人的图片.
	\end{block}
	\begin{columns}
		\column{0.3\textwidth}
		\centering
		\includegraphics[width=0.5\linewidth]{2018-03-12-10-05-13.png}
		\column{0.7\textwidth}
		\centering
		\includegraphics[width=0.72\linewidth]{2018-03-12-09-57-11.png}
	\end{columns}
	\note{
		行人再识别 是图像检索的子问题 , 他是这样的一个过称:

		输入询问图片,如左图所示,在右图测试图像集合中寻找最为匹配的行人图片

		匹配是指达到两点要求:1 两幅图属于同一个人,2 跨摄像头
	}
\end{frame}

\begin{frame}
	{行人再识别定义}
	\begin{block}{行人再识别 (Person Re-identification, \textit{aka.} ReID)}
		输入某一行人的询问图片(Probe), 在图像或者视频集合(gallery)中跨摄像头检索所有包含该行人的图片.
	\end{block}
	\begin{figure}
		\includegraphics[width=0.9\linewidth]{2018-03-11-22-56-05.png}
	\end{figure}
	\note{
		为了说明再识别的困难,我挑选了模型预测的失败案例。

		左侧上方为预测结果,下防为期望的结果。由于询问蹄片特则不明显,测试集合存在大量干扰背景图片,模型的5个预测结果,军错误。背景干扰包括自信车、草地
	}
\end{frame}


\begin{frame}
	{存在的挑战}
	\begin{description}
		\item[Occlusion] \includegraphics[width=0.9\linewidth]{2018-03-12-10-09-03.png}
		\item[Illumination] \includegraphics[width=0.9\linewidth]{2018-03-12-10-10-10.png}
		\item[Pose] \includegraphics[width=0.9\linewidth]{2018-03-12-10-10-18.png}
		\item[Misalign] \includegraphics[width=0.88\linewidth]{2018-03-07-20-16-25.png}
	\end{description}

	\note{
		除背景干扰,
		再识别的难点还有 遮挡、明暗、姿态倒置的行人外观巨大改变,以及由于检测器误差导致的空间适配。
	}
\end{frame}

\section{技术路线与设计方案}

\subsection{注意力机制}

\begin{frame}{动机}
	空间适配导致的失败案例:
	\begin{figure}
		\centering
		\includegraphics[width=\textwidth]{fig/2018-04-18-21-53-15.png}
		% \caption{我们的基准模型在手工标注的数据集和行人检测器自动检测的数据集上的典型样例} 
		\label{fig:label2det}
	\end{figure}
\end{frame}

\begin{frame}{方案}
	提取局部属性特征:
	\begin{figure}
		\centering
		\includegraphics[width=.7\textwidth]{fig/2018-05-11-16-53-10.png}
		% \caption{兼具提取属性特征作用的残差SE模块}
		\label{fig:resse}
	\end{figure}
	局部与全局特征融合:
	\begin{figure}
		\centering
		\includegraphics[width=.92\textwidth]{fig/2018-05-11-16-54-07.png}
		% \caption{中层特征与全局特征的融合方式}
		\label{fig:fusion}
	\end{figure}
\end{frame}

\begin{frame}{实验}
	在数据集CUHK03上的CMC-1,CMC-5,CMC-10性能指标比对:
	\begin{table}
		\centering
		% \caption{在数据集CUHK03上的CMC-1,CMC-5,CMC-10性能指标比对}
		\scalebox{0.9}{
			\begin{tabular}{c|ccc|ccc}
				\hline
				\multirow{2}*{Methods}                     & \multicolumn{3}{c|}{labelled CUHK03} & \multicolumn{3}{c}{detected CUHK03}                                         \\
				\cline{2-7} \cline{2-7}                    & r=1                                  & r=5                                 & r=10    & r=1     & r=5     & r=10    \\ \hline
				KISSME \cite{kissme}                       & 14.17                                & 37.46                               & 52.20   & 11.70   & 33.45   & 45.69   \\
				LMNN \cite{lmnn}                           & 7.29                                 & 19.64                               & 30.74   & 6.25    & 17.87   & 26.60   \\
				LOMO+XQDA \cite{xqda}                      & 52.20                                & 82.23                               & 92.14   & 46.25   & 78.90   & 88.55   \\ \hline
				ImprovedDL \cite{improveddl}               & 54.74                                & 86.50                               & 93.88   & 44.96   & 76.01   & 81.85   \\
				DCSL (with hnm) \cite{yaqing2016semantics} & 80.20                                & 97.73                               & 99.17   & -       & -       & -       \\
				JLML \cite{jlml}                           & 83.20                                & 98.00                               & 99.40   & 80.60   & {96.90} & {98.70} \\
				SC-PPMN (with hnm) \cite{mao2018multi}     & {85.50}                              & {98.20}                             & {99.50} & {80.63} & 95.62   & 98.07   \\ \hline \hline
				TriHard Baseline                           & 84.91                                & 98.35                               & 99.24   & 81.88   & 96.34   & 98.44   \\
				+ SE Attention                             & 87.28                                & 98.50                               & 99.18   & 85.50   & 95.90   & 97.69   \\
				+ Multi-Scale Fusion                       & 88.28                                & 98.44                               & 99.25   & 85.93   & 95.37   & 97.06   \\
				+ Rerank                                   & 96.34                                & 99.24                               & 99.57   & 91.53   & 98.32   & 98.95   \\ \hline
			\end{tabular}
		}
		\label{tab:cuhk03}
	\end{table}

\end{frame}

\begin{frame}{实验}
	在Market1501和DukeMTMC上的CMC-1, mAP性能指标对比:
	\begin{table}
		\centering
		% \caption{在数据集Market1501和DukeMTMC上的CMC-1, mAP性能指标对比}
		\label{tab:market}
		\begin{tabular}{c|cc|cc}
			\hline
			\multirow{2}*{Method}                & \multicolumn{2}{c|}{Market1501} & \multicolumn{2}{c}{DukeMTMC-ReID}                 \\
			\cline{2-5} \cline{2-5}              & r=1                             & mAP                               & r=1   & mAP   \\ \hline
			SomaNet                              & 73.87                           & 47.89                             & 76.70 & 56.80 \\
			PAN                                  & 82.81                           & 63.35                             & 71.59 & 51.51 \\
			TriHard in \cite{hermans2017defense} & 82.99                           & 66.63                             & 73.24 & 54.60 \\
			AWTL                                 & 84.20                           & 68.03                             & 74.23 & 54.97 \\ \hline  \hline
			TriHard Baseline                     & 84.62                           & 68.68                             & 76.32 & 59.57 \\
			+ SE Attention                       & 86.79                           & 70.61                             & 77.29 & 61.12 \\
			+ Multi-Scale Fusion                 & 87.51                           & 72.34                             & 77.12 & 61.64 \\
			+ Rerank                             & 89.19                           & 85.08                             & 79.39 & 72.57 \\ \hline
		\end{tabular}
	\end{table}
\end{frame}

\subsection{度量学习}
\begin{frame}{三元损失}
	\begin{equation}
		\cL_{\textrm{TriHard}} (\mX, \vtheta) = \sum_{a=1}^{N} \max\left(
		0, m +
		{\color{blue}
		\min_{\substack{
				p=1...N \\
				y_a = y_p}
		} D(\vx_a, \vx_p) }
		-
		{ \color{yellow}
		\max_{ \substack{
				n=1...N \\
				y_a \neq y_n }
		} D(\vx_a,\vx_n) }
		\right) \label{eq:trihard}
	\end{equation}
	\begin{figure}
		\centering
		\includegraphics[width=.7\textwidth]{fig/2018-05-19-23-27-39.png}
		% \caption{从距离矩阵学习的角度理解难样本挖掘三元组损失} 
		\label{fig:distmat-tri}
	\end{figure}
\end{frame}

\begin{frame}{动机}
	三种损失监督学习得到的特征的可分性与鉴别性:
	\begin{figure}
		\centering
		\includegraphics[width=1.1\textwidth]{fig/2018-05-16-15-03-54.png}
		% \caption{三种损失函数监督的验证集特征向量和中心向量在2维平面上降维可视化,使用CUHK03数据集和各损失的基准模型得到}
		\label{fig:losses}
	\end{figure}
\end{frame}

\begin{frame}{方案}

	\begin{figure}
		\centering
		\includegraphics[width=.65\textwidth]{fig/2018-05-19-22-25-00.png}
		% \caption{
		% 	基于对比中心损失的行人再识别总体框图
		% }
		\label{fig:dcent}
	\end{figure}

	\begin{figure}
		\centering
		\includegraphics[width=.9\textwidth]{fig/2018-05-19-22-25-11.png}
		% \caption{
		% 	\textbf{左图:}对比中心损失示意图;\textbf{右图:}难样本三元损失函数示意图
		% }
		\label{fig:dcent2}
	\end{figure}
\end{frame}

\begin{frame}{方案}
	\begin{figure}
		\centering
		\includegraphics[width=1\textwidth]{fig/2018-05-19-22-25-11.png}
		% \caption{
		% 	\textbf{左图:}对比中心损失示意图;\textbf{右图:}难样本三元损失函数示意图
		% }
		% \label{fig:dcent2}
	\end{figure}
	\begin{equation}
		\cL_{\textrm{CCent}} (\mX, \mC, \vtheta) = \frac{1}{2}
		\sum_{i=1}^{N}
		\frac{D(
			\vx_{i}, \vc_{y_i}
			)}{
			\displaystyle \lambda_s
			\min_{\substack{
					j=1...M \\
					j \neq y_i }}
			D(
			\vx_{i}, \vc_{j}
			) + 1 }
		-
		\lambda_d \sum_{i=1}^{M}  \min_{j=1...M} D(\vc_i,\vc_j)
		\label{eq:ccent}
	\end{equation}

\end{frame}

\begin{frame}{实验}
	在数据集CUHK03上的CMC-1,CMC-5,CMC-10性能指标比对:
	\begin{table}
		\centering
		% \caption{在数据集CUHK03上的CMC-1,CMC-5,CMC-10性能指标比对}
		\scalebox{0.9}{
		\begin{tabular}{c|ccc|ccc}
			\hline
			\multirow{2}*{Methods}                     &
			\multicolumn{3}{c|}{labelled CUHK03}       &
			\multicolumn{3}{c}{detected CUHK03}                                                                    \\
			\cline{2-7}
			\cline{2-7}
			                                           & r=1     & r=5     & r=10    & r=1     & r=5     & r=10    \\ \hline
			KISSME \cite{kissme}                       & 14.17   & 37.46   & 52.20   & 11.70   & 33.45   & 45.69   \\
			LMNN \cite{lmnn}                           & 7.29    & 19.64   & 30.74   & 6.25    & 17.87   & 26.60   \\
			LOMO+XQDA \cite{xqda}                      & 52.20   & 82.23   & 92.14   & 46.25   & 78.90   & 88.55   \\ \hline
			ImprovedDL \cite{improveddl}               & 54.74   & 86.50   & 93.88   & 44.96   & 76.01   & 81.85   \\
			DCSL (with hnm) \cite{yaqing2016semantics} & 80.20   & 97.73   & 99.17   & -       & -       & -       \\
			JLML \cite{jlml}                           & 83.20   & 98.00   & 99.40   & 80.60   & {96.90} & {98.70} \\
			SC-PPMN (with hnm) \cite{mao2018multi}     & {85.50} & {98.20} & {99.50} & {80.63} & 95.62   & 98.07   \\ \hline
			\hline
			Xent                                       & 76.93   & 95.16   & 97.76   & 71.67   & 91.36   & 95.34   \\
			+LblSmth                                   & 78.64   & 93.96   & 96.78   & 75.87   & 91.53   & 95.15   \\
			+Cent                                      & 82.21   & 95.55   & 98.39   & 80.12   & 95.38   & 98.11   \\
			% CCent                                      & 17.51   & 43.11   & 59.72   &         &         &         \\ 	 
			TriHard                                    & 84.50   & 98.49   & 99.24   & 82.48   & 97.15   & 98.38   \\
			+CCent                                     & 87.77   & 98.70   & 99.55   & 84.86   & 97.61   & 98.48   \\
			+CCent+Dis                                 & 88.12   & 98.80   & 99.40   & 84.45   & 97.15   & 98.44   \\   \hline
			\hline
		\end{tabular}
		}
		\label{tab:cuhk032}
	\end{table}
\end{frame}

\begin{frame}{实验}
	在Market1501和DukeMTMC上的CMC-1, mAP性能指标对比:
	\begin{table}
		\centering
		% \caption{在数据集Market1501和DukeMTMC上的CMC-1, mAP性能指标对比}
		\label{tab:market2}
		\begin{tabular}{c|cc|cc}
			\hline
			\multirow{2}*{Method}                & \multicolumn{2}{c|}{Market1501} & \multicolumn{2}{c}{DukeMTMC-ReID}                 \\
			\cline{2-5} \cline{2-5}              & r=1                             & mAP                               & r=1   & mAP   \\ \hline
			SomaNet \cite{zheng2017ped}            & 73.87 & 47.89           & 76.70 & 56.80 \\
			PAN \cite{barbosa2017looking}          & 82.81 & 63.35             & 71.59 & 51.51 \\
			TriHard in \cite{hermans2017defense}   & 82.99 & 66.63       & 73.24 & 54.60 \\
			AWTL        \cite{ristani2018features} & 84.20 & 68.03          & 74.23 & 54.97 \\ \hline  \hline
			Xent                                   & 84.92 & 65.15         & 72.44 & 51.16 \\
			+LblSmth                               & 89.46 & 72.18        & 79.08 & 59.80 \\
			+Cent                                  & 82.42 & 61.77          & 73.81 & 53.29 \\
			TriHard                                & 86.55 & 69.98         & 75.08 & 57.10 \\
			+CCent                                 & 89.12 & 71.57         & 81.54 & 62.33 \\
			+CCent+Dis                             & 90.37 & 72.94       & 80.36 & 60.87 \\  \hline
		\end{tabular}
	\end{table}
\end{frame}

\section{思考与展望}

\begin{frame}{不变性与同变性}
	\begin{figure}
		\centering 
		\includegraphics[width=.75\textwidth]{2018-05-21-10-51-51.png}
	\end{figure}
\end{frame}

\begin{frame}{显著性与可解释性}
	\begin{figure}
		\centering 
		\includegraphics[width=.47\textwidth]{2018-05-21-09-58-23.png} 
		\includegraphics[width=.47\textwidth]{2018-05-21-10-31-13.png}
	\end{figure}
	% \begin{figure}
	% 	\centering 
	% \end{figure}
\end{frame}

\begin{frame}[t, allowframebreaks]
	\frametitle{参考文献}
	\printbibliography
\end{frame}

\begin{frame}
	\chuhao Thank you! %\fontspec{LHANDW.TTF}
\end{frame}


\begin{frame}[fragile]
	\frametitle{eval protocol}
	\begin{lstlisting}
cmc_configs = {
'cuhk03': dict(separate_camera_set=True,
  single_gallery_shot=True,
  first_match_break=False),
'market1501': dict(separate_camera_set=False,#h
  single_gallery_shot=False,  # hard
  first_match_break=True),
'allshots': dict(separate_camera_set=False,#h
  single_gallery_shot=False,  # hard
  first_match_break=False),
}
\end{lstlisting}
\end{frame}

\end{document}
